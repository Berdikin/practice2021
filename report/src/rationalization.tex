\begin{center}
\bfseries{\large МАТЕРИАЛЫ ПО РАЦИОНАЛИЗАТОРСКИМ ПРЕДЛОЖЕНИЯМ}
\end{center}

Так как в данный момент функция распознавания лиц всё ещё имеет не самую низкую вероятность ошибиться из-за того, что у разных фотографий в базе данных разное освещение, качество и т.д., необходимо качество распознавания улучшать. Я делал это путём добавления в базу данных дополнительные фотографии одних и тех же людей под тем же именем. Конечно, у такого подхода есть и большой минус: при неправильном определении лица шанс на на правильное распознавание уменьшится, а не увеличится.

Для увеличения точности распознавания лица было бы верным решением автоматизировать редактирование базы данных, поскольку сейчас это возможно сделать только вручную.

\pagebreak
